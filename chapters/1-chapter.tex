% ----------------------------------------------------------
\chapter{Visão Geral do Projeto}
% ----------------------------------------------------------

Nos dias de hoje, um refrigerador é um item indispensável em uma residência. Porém, a sua popularização é relativamente recente, datando da década de 1920, quando os refrigeradores elétricos passaram a ser acessíveis à população. No Brasil, o primeiro refrigerador nacional foi criado em Joinville, Santa Catarina, pela Consul, empresa criada no ano de 1950. Segundo Ono (2004), o primeiro modelo lançado foi o Q-300 em 1950, com um sistema de refrigeração por absorção, funcionando à base de querosene. Em 1956, a Consul também foi pioneira em trazer para o país o primeiro refrigerador elétrico, o Consul Elétrico.

Dada sua ampla utilização, os refrigeradores domésticos respondem por uma parcela significativa do consumo de energia. Juntamente com condicionadores de ar, esses dispositivos respondem por cerca de 17,2\% de todo o consumo de energia mundial (COULOMB et al., 2015). De acordo com  estimativas, existem cerca de 1,5 bilhões de refrigeradores domésticos em atividade no mundo \cite{refrigeration-ademe},.

Segundo o Anuário Estatístico de Energia Elétrica de 2020, da EMPRESA BRASILEIRA DE PESQUISA ENERGÉTICA (2020, p 48), o consumo de energia elétrica no setor residencial do Brasil passou de 131 190 GWh em 2015 para 142 781 GWh em 2019, o que representa um aumento de, aproximadamente, 8\% em quatro anos. Com o aumento do consumo de energia elétrica, surge a necessidade da fabricação de produtos que consumam menos energia e sejam mais eficientes, além de serem ecologicamente aceitáveis. Dentro dessa problemática, os compressores utilizados em refrigeradores domésticos são constantemente vistos como objetos de estudo em pesquisas relacionadas à eficiência energética.

Compressores alternativos, que geralmente realizam o processo de compressão de vapor através do movimento de um mecanismo biela-manivela, são os mais aplicados na refrigeração doméstica. Esse tipo de compressor possui diversos fatores que influenciam na perda de eficiência. Segundo Pérez-Segarra et al. (2005, apud SILVA E DUTRA, 2020, p. 5), as perdas de energia em um compressor podem ser divididas em três partes: (i) perdas elétricas causadas pela resistência ôhmica no motor, (ii) perdas mecânicas geradas pela fricção nos mancais e (iii) perdas termodinâmicas causadas pelas irreversibilidades durante o processo de compressão do gás.

Compressores lineares são uma alternativa interessante aos compressores alternativos. O modelo linear não utiliza o movimento rotativo dos motores elétricos tradicionais, mas usa um motor elétrico linear para promover o movimento alternado do pistão. O funcionamento de um motor linear é semelhante, porém o estator, que é onde as bobinas estão, não é circundado ao rotor, nesse caso conhecido como cursor. Em outras palavras, o cursor se move axialmente em relação ao estator. Complementando o que foi dito, Liang (2017) descreve:

\begin{citacao}
Compressor linear não possui mecanismo biela-manivela quando comparado com compressores convencionais alternativos. Isso permite maior eficiência,  ausência de óleo lubrificante, menor custo e menor tamanho quando comparado com modelos comuns usados em sistemas de refrigeração por compressão de vapor.(LIANG, 2017, p. 253, tradução nossa)
\end{citacao}

Seguindo a ideia de Liang (2017), um compressor linear não possui o mecanismo de biela-manivela, então o atrito com os mancais é desconsiderado, além disso, não existe a necessidade de óleo lubrificante, tornando assim o sistema mais atrativo em relação ao cuidado com o meio ambiente. Dessa forma, o desenvolvimento da tecnologia dos compressores lineares se torna pertinente, visto que cumpre as necessidades de ser um produto energeticamente mais eficiente e menos agressivo ao meio ambiente.

Compressores lineares já são empregados em alguns refrigeradores domésticos disponíveis no mercado. Desde 2002 a empresa sul-coreana LG tem comercializado este tipo de compressor para sistemas de refrigeração, tendo licenciado a tecnologia da empresa Sunpower. No Brasil, a Embraco desenvolveu a tecnologia Wisemotion, que é uma linha de compressores lineares que não utiliza óleo, tendo sido lançada no mercado em 2014. Tronbini (2016) explica algumas qualidades oferecidas pela tecnologia:

\begin{citacao}
A tecnologia atende níveis de eficiência que podem chegar a 20\% de economia de energia quando comparada aos compressores de alta eficiência mais vendidos mundialmente. “Outra vantagem do Wisemotion é proporcionar uma melhor conservação dos alimentos, por manter a temperatura interna do refrigerador estável”, aponta a empresa. O desenvolvimento do compressor levou dez anos e envolveu 100 profissionais de Pesquisa e Desenvolvimento, tendo gerado 80 patentes globalmente. (TRONBINI, 2016, online)
\end{citacao}

A partir das informações apresentadas, é visível a inovação que os compressores lineares trazem em relação aos alternativos convencionais, se consolidando como uma alternativa tecnológica para aumentar a eficiência energética de refrigeradores domésticos. Entretanto, sendo esta uma tecnologia recente, ainda apresenta grandes oportunidades de melhoria. Um exemplo disso é a proposta de Silva e Dutra (2020) de otimizar a trajetória do pistão ao longo do processo de compressão, o que tem potencial de aumentar a eficiência termodinâmica em até 25,9\% e a eficiência volumétrica em até 6,1\% em condições típicas de refrigeração doméstica. Dentro deste contexto, espera-se neste trabalho estudar e implementar o modelo eletromecânico de um compressor linear a fim de permitir avaliar futuramente as perdas elétricas e mecânicas de um compressor com trajetória de pistão otimizada.


%As orientações aqui apresentadas são baseadas em um conjunto de normas elaboradas pela \gls{ABNT}. Além das normas técnicas, a Biblioteca também elaborou uma série de tutoriais, guias, \textit{templates} os quais estão disponíveis em seu site, no endereço \url{http://portal.bu.ufsc.br/normalizacao/}.

%Paralelamente ao uso deste \textit{template} recomenda-se que seja utilizado o \textbf{Tutorial de Trabalhos Acadêmicos} (disponível neste link \url{https://repositorio.ufsc.br/handle/123456789/180829}) e/ou que o discente \textbf{participe das capacitações oferecidas da Biblioteca Universitária da UFSC}.

%Este \textit{template} está configurado apenas para a impressão utilizando o anverso das folhas, caso você queira imprimir usando a frente e o verso, acrescente a opção \textit{openright} e mude de \textit{oneside} para \textit{twoside} nas configurações da classe \textit{abntex2} no início do arquivo principal \textit{main.tex} \cite{abntex2classe}.

%Os trabalhos de conclusão de curso (TCC) de graduação e de especialização não são entregues em formato impresso na Biblioteca Universitária. Porém, sua versão PDF pode ser disponibilizada no Repositório Institucional, consulte seu curso sobre os procedimentos adotados para a entrega. 

% ----------------------------------------------------------
%Uma nota de rodapé, já tem seu estilo automático com o comando \texttt{$\backslash$footnote}\footnote{As notas de rodapé possuem fonte tamanho 10. O alinhamento das linhas da nota de rodapé deve ser abaixo da primeira letra da primeira palavra da nota de modo dar destaque ao expoente.}.


% ----------------------------------------------------------
\section{Objetivos}
% ----------------------------------------------------------

Para desenvolver este trabalho, propõe-se os seguintes objetivos.

% ----------------------------------------------------------
\subsection{Objetivo Geral}
% ----------------------------------------------------------

Implementar um modelo numérico do sistema eletromecânico de um compressor linear.

% ----------------------------------------------------------
\subsection{Objetivos Específicos}
% ----------------------------------------------------------

\begin{itemize}
\item
Implementar e validar um submodelo elétrico de motor linear.
\item
Implementar e validar um submodelo mecânico para o movimento do pistão.
\item
Implementar um submodelo termodinâmico simplificado para o ciclo de compressão.
\item
Acoplar os submodelos elétrico, mecânico e termodinâmico em um único modelo.
\item
Comparar  os resultados do modelo implementado com dados da literatura.
\item
Avaliar a influência de diferentes parâmetros sobre o funcionamento do compressor.
\end{itemize}